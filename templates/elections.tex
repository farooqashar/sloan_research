\documentclass{article}


\usepackage{arxiv}

\usepackage[utf8]{inputenc} % allow utf-8 input
\usepackage[T1]{fontenc}    % use 8-bit T1 fonts
\usepackage{hyperref}       % hyperlinks
\usepackage{url}            % simple URL typesetting
\usepackage{booktabs}       % professional-quality tables
\usepackage{amsfonts}       % blackboard math symbols
\usepackage{nicefrac}       % compact symbols for 1/2, etc.
\usepackage{microtype}      % microtypography
\usepackage{lipsum}
\usepackage{graphicx} %package to manage images
\graphicspath{ {./images/} }

\usepackage[rightcaption]{sidecap}

\usepackage{wrapfig}
\title{Elections and Social Media Influence}


\author{
  Ashar Farooq\\
  Department of Electrical Engineering and Computer Science\\
  Massachusetts Institute of Technology\\
  Cambridge, MA 02139 \\
  \texttt{afarooq@mit.edu} \\
  %% \AND
  %% Coauthor \\
  %% Affiliation \\
  %% Address \\
  %% \texttt{email} \\
  %% \And
  %% Coauthor \\
  %% Affiliation \\
  %% Address \\
  %% \texttt{email} \\
  %% \And
  %% Coauthor \\
  %% Affiliation \\
  %% Address \\
  %% \texttt{email} \\
}

\begin{document}
\maketitle
\begin{abstract}
An election is the framework for a functioning and effective democracy. Any threat on this critical infrastructure is a threat on the ideals of democracy, republicanism, freedom, and civil liberties. To this effect, it is important that governments and societies do their due diligence to maintain the integrity, effectiveness, and trustworthiness of elections. 

\vspace{5mm} %5mm vertical space

While there are many threats to elections in existence, including inadequate voting machines and lack of quality voter registration systems, the largest, most relevant, and complex issue that undermines the integrity of an election is social media influence. With rapid technological advances happening in the 21st century, the abuse of social media and virtual platforms is ever increasing. When these technologies are used by the right people for the right purposes, they are the best that the human civilization has to offer. However, when they are abused by certain adversaries, both foreign and domestic, for inappropriate purposes, they hurt the integrity of a democratic election and the human way of living. 

\vspace{5mm} %5mm vertical space

The widespread influence of social media platforms is clearly visible in today's world. Individuals around the globe are interconnected as news, both good and bad, spreads like wildfire. Take this widespread influence to the highly divisive political system of the United States and an entire field of election interference via social media influence is born. Election interference is not new; rather, the method of election interference is adapting to the technological realities of the modern world. In a way, these online advances are making it much easier for adversaries to interfere with the democratic process as posting false information on these large social media accounts is much easier and much more effective than tampering with a physical voting machine in a certain polling place. 

\vspace{5mm} %5mm vertical space

This ease of election interference via social media was unambiguously seen in the 2016 United States Presidential Election between Donald Trump and Hillary Clinton. With a seemingly close election and a hotly chaotic election climate, it only takes some social media tide to swing the perspectives of undecided voters in some key battleground states in order to change the outcome of the entire presidential election. Combine this chaotic election climate with partisan politics, the wide popularity/usage of social media platforms, and foreign preference for certain United States politicians/policies to get the Russian Foreign Interference in the 2016 Election in order to elect Donald Trump as the president. This case highlights the immense power of adversaries using social media and its sheer incredible effectiveness. 

\vspace{5mm} %5mm vertical space

The foreign interference in the election was coordinated incredibly well by the Russian Internet Research Agency(IRA) using online manipulation and disinformation propaganda on social media platforms like Facebook in order to ruin the fabric of a fair election. This information warfare regime by Russia ultimately was aimed at changing the perspectives and votes of voters in key states. When these voters utilized social media, they were prone to seeing certain tailored disinformation and fake news designed to influence their minds and votes. The Russian troll farms and disinformation specialists utilized the simple practices of using a fake social media account, supporting certain political issues important to certain Trump supporters, and spreading fake stories about election practices and views of Clinton. The Russian campaign also sought to utilize the power of social media platforms to suppress voter turnout from key demographics by posting and spreading fake information about election practices, locations, and other details. 

\vspace{5mm} %5mm vertical space

The 2016 election brings about many concerns of social media influence on domestic future elections. One of these concerns going forward is that the fast velocity of fake information on social media makes it difficult for social media companies to manage content and promote free expression of political speech while enforcing their policies on fake news/extreme social media influence. This is a complicated balance to perfectly strike because social media influence is only becoming more and more common very recently, making it hard to keep up with the incredible pace. Various abuses of social media platforms by domestic adversaries and lack of regulations on these critical issues of social media influence in elections combine to create many emerging challenges for future elections and for a sustainable democratic model of government in the United States. 

\vspace{5mm} %5mm vertical space

One critical challenge in future elections will be the increased prevalence of domestic election interference via social media. The 2016 Russian meddling in the election was mainly an issue that was centered on foreign election interference, which is governed under additional legal and humanitarian guidelines and regulations. However, domestic election interference by American citizens via social media is one that is very hard to enforce under existing legal framework. In fact, the free expression of speech on social media platforms in the United States is in conflict with social media companies trying to regulate their platforms for disinformation and influence. These large social media companies are often hesitant to strictly enforce regulations relating to domestic Americans as opposed to foreign adversaries. 


\vspace{5mm} %5mm vertical space

Regardless, more research and discussions should be conducted on potential government response, company self-regulation, or mutual partnership for reducing election interference via social media influence. Other topics for further consideration include micro-targeting, usage of political advertisements, verification of accounts, content management, deep fakes, and future of politics in the digital age.

\end{abstract}


\end{document}



\begin{center}
  \url{https://www.ctan.org/pkg/booktabs}
\end{center}
See Figure \ref{fig:fig1}. Here is how you add footnotes. \footnote{Sample of the first footnote.}


